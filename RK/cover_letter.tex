%% start of file `template.tex'.
%% Copyright 2006-2013 Xavier Danaux (xdanaux@gmail.com).
%
% This work may be distributed and/or modified under the
% conditions of the LaTeX Project Public License version 1.3c,
% available at http://www.latex-project.org/lppl/.


\documentclass[10pt,a4paper,sans]{moderncv}        % possible options include font size ('10pt', '11pt' and '12pt'), paper size ('a4paper', 'letterpaper', 'a5paper', 'legalpaper', 'executivepaper' and 'landscape') and font family ('sans' and 'roman')

% moderncv themes
\moderncvstyle{classic}                            % style options are 'casual' (default), 'classic', 'oldstyle' and 'banking'
\moderncvcolor{green}                              % color options 'blue' (default), 'orange', 'green', 'red', 'purple', 'grey' and 'black'
%\renewcommand{\familydefault}{\sfdefault}         % to set the default font; use '\sfdefault' for the default sans serif font, '\rmdefault' for the default roman one, or any tex font name
%\nopagenumbers{}                                  % uncomment to suppress automatic page numbering for CVs longer than one page

% character encoding
\usepackage[utf8]{inputenc}                       % if you are not using xelatex ou lualatex, replace by the encoding you are using
%\usepackage{CJKutf8}                              % if you need to use CJK to typeset your resume in Chinese, Japanese or Korean
\usepackage{url}
% adjust the page margins
\usepackage[scale=0.75]{geometry}
%\setlength{\hintscolumnwidth}{3cm}                % if you want to change the width of the column with the dates
%\setlength{\makecvtitlenamewidth}{10cm}           % for the 'classic' style, if you want to force the width allocated to your name and avoid line breaks. be careful though, the length is normally calculated to avoid any overlap with your personal info; use this at your own typographical risks...

% personal data
\name{Qingpeng}{Zhang}
%\title{Resumé title}                               % optional, remove / comment the line if not wanted
\address{Department of Computer Science and Engineering}{Michigan State University}{East Lansing, MI 48824}% optional, remove / comment the line if not wanted; the "postcode city" and and "country" arguments can be omitted or provided empty
\phone[mobile]{+1~(678)~901~9911}                   % optional, remove / comment the line if not wanted
%\phone[fixed]{+2~(345)~678~901}                    % optional, remove / comment the line if not wanted
%\phone[fax]{+3~(456)~789~012}                      % optional, remove / comment the line if not wanted
\email{qingpeng@msu.edu}                               % optional, remove / comment the line if not wanted
%\homepage{www.johndoe.com}                         % optional, remove / comment the line if not wanted
%\extrainfo{additional information}                 % optional, remove / comment the line if not wanted
%\photo[64pt][0pt]{picture.png}                       % optional, remove / comment the line if not wanted; '64pt' is the height the picture must be resized to, 0.4pt is the thickness of the frame around it (put it to 0pt for no frame) and 'picture' is the name of the picture file
\quote{Some quote}                                 % optional, remove / comment the line if not wanted

% to show numerical labels in the bibliography (default is to show no labels); only useful if you make citations in your resume
%\makeatletter
%\renewcommand*{\bibliographyitemlabel}{\@biblabel{\arabic{enumiv}}}
%\makeatother
%\renewcommand*{\bibliographyitemlabel}{[\arabic{enumiv}]}% CONSIDER REPLACING THE ABOVE BY THIS

% bibliography with mutiple entries
%\usepackage{multibib}
%\newcites{book,misc}{{Books},{Others}}
%----------------------------------------------------------------------------------
%            content
%----------------------------------------------------------------------------------
\begin{document}
%-----       letter       ---------------------------------------------------------
% recipient data
\recipient{Department of Computer Science and Engineering}{University of California, San Diego\\9500 Gilman Drive\\La Jolla, CA 92093}
\date{February 02, 2015}
\opening{Dear Dr. Knight,}
\closing{Sincerely,}
\enclosure[Attached]{curriculum vit, research statement{}}          % use an optional argument to use a string other than "Enclosure", or redefine \enclname
\makelettertitle

I am excited to apply for the postdoc position in your group.  I am strongly interested in developing 
and applying computational methods to guide large scale efforts of using sequencing technologies 
as a tool to answer biological questions.  I have more than ten years of experience in bioinformatics 
and have been working with metagenomic data as part of my PhD.  I specialize in developing 
scalable tools for processing and analyzing complex shotgun metagenomes. Currently, I am a 
graduate student in Dr. C. Titus Brown's group at Michigan State University and expect to receive 
my doctorate this May.  

In my years of PhD research working on metagenomic data, I have come across many great papers 
from your group, from the earlier one about the clustering metric  UniFrac  to the recent review  on 
personal human microbiome.  The tools developed in your group and your insights to the future of 
microbiome research benefit my work significantly.  For example, QIIME is integrated into the 
pipeline of my recent work on efficient diversity analysis, using biom format to store the IGS tables. 
The package scikit-bio is also heavily used to facilitate  the analysis. I even happened to come across 
a problem and submitted a bug ticket to the developers.  

Based on my previous work on an efficient k-mer counting approach and digital normalization, 
recently I am working on a novel method to enable efficient and scalable microbial diversity 
analysis  without the requirement of assembly and reference sequences. A novel concept - IGS
 (informative genomic segment) is proposed to represent the unique information in a metagenomics 
 data set.  The abundance of IGSs in different samples can be retrieved by mapping the reads to 
 the de Bruijn graph database built from separate samples. After we get the samples-by-IGS 
 matrix, existing package like QIIME can be used to do different kinds of diversity analysis. I have 
 evaluated this method on multiple metagenomes from a variety of environments (e.g., human 
 microbiome,  soil in collaboration with James Tiedje, ballast water viromes in 
 collaboration with Joan Rose ). I believe that the IGSs can be used as a cornerstone for diversity 
 analysis of whole shotgun metagenomics data sets just like OTU for 16S rRNA data sets. The new assembly-free, 
 reference-free framework to do diversity analysis based on IGS will be a beneficial complement 
 to the QIIME package, which is mainly focused on analysis based on OTU.  Also, there is more 
 work to do to tackle the computational challenge to accommodate the much larger and more 
 complicated samples-by-IGS matrix, including, but not limited to, adding new feature to QIIME, 
 biom and/or sickit-bio.   Going forward, I plan to integrate those methods I have been working 
 on to solve more problems in metagenomics , like binning approaches, functional annotation, 
 and phylogenetic analysis, with the adoption of more machine learning methods, data structures 
 or algorithms while necessary.

About the future of microbiome science I come up with many ideas previously. But I read the news 
about  2016 budget submitted by the president today and it can not be better to cite several numbers
 from it to emphasize my points. Firstly \$200 million will be allotted to the precision medicine 
 initiative. According to a director in the White House, "precision medicine" is a term for " tailoring 
 treatments to an individual's genetic makeup, microbiome, and other factors"  So obviously 
 microbiome science research will play an important role in facilitating precision medicine or 
 personalized medicine.  Also \$130 million will be allotted to an effort to create a research cohort 
 consisting 1-million-volunteers. This reminds me of Human Microbiome Project, American Gut 
 Project and the biobank effort in San Diego. Another interesting number is that the biggest bump 
 within DOE goes to their Advanced Scientific Computing Research program, which would see an 
 increase of 14.8\%, to \$621 million. This may not be directly related to microbiome science, but 
 it emphasizes the crucial role computing will play to advance scientific research, including microbiome
 science research. To analyze and  interpret the big biological data like that from American Gut  Project 
  or the 1-million-volunteers cohort and make sense of 
 the variations and patterns inside, powerful, efficient, scalable computational methods will be 
 highly demanded. Admittedly the numbers on the budget request may not be approved in the end. 
 But they  can still give some hints about the future of microbiome science in some sort, 
 which is to improve the knowledge about the microbial world and address the problems in 
 precision medicine or personalized medicine to benefit human welfare, assisted by the 
 development of more advanced computation.
 
As discussed above, this is great times to work on microbiome science with the help of efficient 
and powerful computing. I have an extensive history of working with sequencing data (since 2003). 
 Previously, I worked at the Beijing Genomics Institute (BGI) where I participated in genome 
 sequencing projects and later on systems biology such as regulatory network analysis and 
 microRNA prediction using Bayesian classifier. These experiences, combined with my PhD research and my dual major
  in computer science and quantitative biology, have given me a strong background in bioinformatics, 
  molecular biology, and genomics.  I enjoy interacting with a variety of collaborators, including 
  biologists, statisticians, bioinformaticians, and software engineers. 
 
With your supervision and the collaboration with members in your group , I believe I can 
contributing my skills and knowledge to facilitate the microbiome  research, and make a 
difference in a broader field like precision medicine subsequently. I am greatly enthusiastic about
 the possibility of working in your group. I have attached my CV and 
research statement for your review. I would greatly appreciate the opportunity to talk to you 
more about the position. Please do not hesitate to contact me with any questions.



\makeletterclosing

\end{document}


%% end of file `template.tex'.
